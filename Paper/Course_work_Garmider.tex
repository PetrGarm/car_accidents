\documentclass[a4paper, 14pt]{article}
\usepackage[utf8]{inputenc}
\usepackage{amsmath,amsfonts,amssymb,amsthm,mathtools} % AMS
\usepackage{wrapfig,lipsum, cleveref}
\usepackage{icomma} 
\usepackage{color}
\usepackage{geometry} 
\usepackage{longtable}
\usepackage{booktabs}

\linespread{1.5}

\geometry{top=25mm}
\geometry{bottom=35mm}
\geometry{left=35mm}
\geometry{right=20mm}


%% Номера формул
%\mathtoolsset{showonlyrefs=true} % Показывать номера только у тех формул, на которые есть \eqref{} в тексте.

%% Шрифты
\usepackage{euscript}	 % Шрифт Евклид
\usepackage{mathrsfs} % Красивый матшрифт

%% Свои команды
\DeclareMathOperator{\sgn}{\mathop{sgn}}

%% Перенос знаков в формулах (по Львовскому)
\newcommand*{\hm}[1]{#1\nobreak\discretionary{}
{\hbox{$\mathsurround=0pt #1$}}{}}


\title{Вариация алгоритма кросс-валидации со взвешиванием наблюдений}
\usepackage{cmap}					% поиск в PDF
\usepackage[T2A]{fontenc}			% кодировка
\usepackage[utf8]{inputenc}			% кодировка исходного текста
\usepackage[english,russian]{babel}	% локализация и переносы
\usepackage{graphicx}
\graphicspath{{pictures/}}
\DeclareGraphicsExtensions{.pdf,.png,.jpg}
\author{Гармидер Петр}
\date{\today}
\begin{document}


\thispagestyle{empty}
\begin{center}
	\textbf{ПРАВИТЕЛЬСТВО РОССИЙСКОЙ ФЕДЕРАЦИИ}\\
	\vspace{2ex}
	\textbf{Федеральное государственное автономное образовательное учреждение \\ высшего образования Национальный исследовательский университет \\ <<Высшая школа экономики>>}
	
	
	\vspace{8ex}
	
	\textbf{Факультет компьютерных наук}
\end{center}
\vspace{9ex}

\begin{center}
	{\textbf{КУРСОВАЯ РАБОТА
	}}
	\vspace{1ex}
	
	\underline{БАЙЕСОВСКИЙ ПОДХОД ДЛЯ АНАЛИЗА ДТП} \\
	\underline{BAYESIAN APPROACH TO CAR ACCIDENTS ANALYSIS}\\
	\vspace{1.5ex}
	
	по направлению подготовки \underline{01.04.02 Прикладная математика и информатика} \\
	образовательная программа \underline{<<Науки о Данных>>}
	
\end{center}
\vspace{1ex}
\begin{flushright}
	\noindent
	Студент группы АИД-Б20:\\Гармидер Петр Александрович\\
	\vspace{13ex}
	Научный руководитель:\\
	Борис Демешев
	
\end{flushright}	

\vfill

\begin{center}
	Москва 2021
	
\end{center}
\newpage

\tableofcontents

\newpage

\section{Введение}
\subsection{Актуальность в РФ} 
Транспорт является важной экономической и социальной составляющей жизни населения. Безопасность и эффективность транспортного передвижения непосредственно влияют на качество жизни. Пусть юридически, дороги могут находиться не только в государственной собственности, но и принадлежать частным, юридическим лицам, но по факту за качество большей части дорог отвечает государство. Притом, у государства есть множество вариантов воздействия на дорожное движение: качество и расположение построенных дорог, инфраструктура прилагаемая к дорогам, устанавливаемые скоростные режими и т.д.  


\newpage
\bibliographystyle{utf8gost705u}  %% стилевой файл для оформления по ГОСТу
\bibliography{biblio}     %% имя библиографической базы (bib-файла) 




\end{document}
